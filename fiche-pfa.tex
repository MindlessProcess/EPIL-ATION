\documentclass[12pt,a4paper]{article}
\title{\textbf{EPITECH PFA - EPIL}(ATION)\\
  \textbf{Easy Programming Interaction Library}\\
  (And Transformation Interface On 'Nux)}

\author{Lyoma GUILLOU, Lucas MERLETTE}
\date{21/01/2014}

\begin{document}
\maketitle

\newpage
\section{Introduction}
Le projet EPIL est une librairie qui permet \`{a} un processus de faciliter la modification de son code source ainsi que de le relancer en une nouvelle instance.

\section{Contexte}
Le programme impl\'{e}mentant EPIL dresse une liste d'\'{e}v\'{e}nements dont il souhaite modifier le comportement.
Lorsqu'un \'{e}v\'{e}nement est rencontr\'{e}, le programme indique \`{a} EPIL quelles actions mener : modification du code source, recompilation du programme, relancement du programme.\\
Les actions peuvent \^{e}tre men\'{e}es ind\'{e}pendament des autres et dans n'importe quel ordre, ce qui laisse \`{a} l'utilisateur le choix de l'ex\'{e}cution de son programme.

\newpage
\section{Equipe}
Lyoma GUILLOU - Chef de Projet\\
Lucas MERLETTE - Concepteur g\'{e}n\'{e}ral

\section{Objectifs}
Les objectifs fix\'{e}s pour notre programme principal sont:
\\- L'\'{e}laboration d'une couche de communication entre la librairie EPIL et le programme, symbolis\'{e}e par des fonctions.
\\- Le d\'{e}veloppement de fonctions appliquant les fonctionnalit\'{e}s de EPIL: modification du code, recompilion du programme, relancement du programme.

\section{Planning General}

\begin{enumerate}
\item Lun. 27 Jan. 2014 - Conception, Brainstorm
\begin{itemize}
\item[-] Venir avec un plan (Lyoma, Lucas)
\end{itemize}
\item Sam. 15 F\'{e}v. 2014 - Couche communication \'{e}labor\'{e}e
\begin{itemize}
\item[-] Classes Epil et Profile (Lucas)
\item[-] Classe Range (Lyoma)
\end{itemize}
\item Sam. 08 Mar. 2014 - Fonctionnalit\'{e}s de EPIL (voir Objectifs) d\'{e}velopp\'{e}es
\begin{itemize}
\item[-] Classe System : liaison avec la couche de communication (Lucas)
\item[-] Classes Action et System : liaison avec Action  (Lyoma)
\end{itemize}
\item Sam. 29 Mar. 2014 - Ensemble d'applications d\'{e}monstratives
\begin{itemize}
\item[-] D\'{e}velopper des applications de test (Lyoma, Lucas)
\item[-] R\'{e}diger la documentation (Lucas)
\end{itemize}
\end{enumerate}

\end{document}
